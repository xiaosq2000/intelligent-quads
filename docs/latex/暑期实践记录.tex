\documentclass{tufte-handout}
\title{暑期实践日志}
\author{肖书奇\quad 指导教师:熊昊}
\date{} % This line cancels today's date
\usepackage{ifxetex}% For compatibility with xelatex engine
\ifxetex
\newcommand{\textls}[2][5]{%
    \begingroup\addfontfeatures{LetterSpace=#1}#2\endgroup
}
\renewcommand{\allcapsspacing}[1]{\textls[15]{#1}}
\renewcommand{\smallcapsspacing}[1]{\textls[10]{#1}}
\renewcommand{\allcaps}[1]{\textls[15]{\MakeTextUppercase{#1}}}
\renewcommand{\smallcaps}[1]{\smallcapsspacing{\scshape\MakeTextLowercase{#1}}}
\renewcommand{\textsc}[1]{\smallcapsspacing{\textsmallcaps{#1}}}
\usepackage[osf,sc]{mathpazo} % tufte style font
\usepackage[scaled=0.90]{helvet} % tufte style font
\usepackage[scaled=0.85]{beramono} % tufte style font
\fi
\usepackage{ctex} % 支持中文
\usepackage{xeCJK}% 支持中文
\xeCJKsetup{PunctStyle=kaiming}  % 设置中文标点符号,开明式
% \xeCJKsetup{PunctStyle=quanjiao} % 设置中文标点符号,全角式
% \setCJKmainfont{仓耳今楷05-6763}      % 设置正文罗马族的 CJK 字体,影响 \rmfamily 和 \textrm 的字体。
%     \setCJKmainfont{STZhongsong}      % 设置正文罗马族的 CJK 字体,影响 \rmfamily 和 \textrm 的字体。
%     \setCJKsansfont{STZhongsong}  % 设置正文无衬线族的 CJK 字体,影响 \sffamily 和 \textsf 的字体。
%     \setCJKmonofont{SimHei}          % 设置正文等宽族的 CJK 字体,影响 \ttfamily 和 \texttt 的字体。
\usepackage{amsmath} % For AMS math support
\usepackage{amssymb} % For AMS math support
\usepackage{amsfonts} % For AMS math support
\usepackage{booktabs} % For nicely typeset tabular material
\usepackage{makecell} % For multiple lines in one cell in table environment
\usepackage{graphicx} % For graphics / images
\setkeys{Gin}{width=\linewidth,totalheight=\textheight,keepaspectratio} % Set default scale globally
\graphicspath{{graphics/}} % Set paths to search for images
% \usepackage{makeidx} % For index
% \makeindex
\usepackage{romannum} % For Roman numbers
\usepackage[dvipsnames]{xcolor} % For various colors
\usepackage{listings}
\lstset{
    % basicstyle=\footnotesize\ttfamily,
        basicstyle=\ttfamily,
        breaklines=true,
        columns=fixed,       
        % numbers=left,                                        % 在左侧显示行号
            frame=none,                                          % 不显示背景边框
            % backgroundcolor=\color[RGB]{250,250,250},            % 设定背景颜色
            % keywordstyle=\color[RGB]{40,40,255},                 % 设定关键字颜色
            % numberstyle=\footnotesize\color{darkgray},           % 设定行号格式
            % commentstyle=\it\color[RGB]{0,96,96},                % 设置代码注释的格式
            % stringstyle=\rmfamily\slshape\color[RGB]{128,0,0},   % 设置字符串格式
            showstringspaces=false,                              % 不显示字符串中的空格
            % language=[11]c++,                                        % 设置语言
            % morekeywords={alignas,continute,friend,register,true,alignof,decltype,goto,
                % reinterpret_cast,try,asm,defult,if,return,typedef,auto,delete,inline,short,
                % typeid,bool,do,int,signed,typename,break,double,long,sizeof,union,case,
                % dynamic_cast,mutable,static,unsigned,catch,else,namespace,static_assert,using,
                % char,enum,new,static_cast,virtual,char16_t,char32_t,explict,noexcept,struct,
                % void,export,nullptr,switch,volatile,class,extern,operator,template,wchar_t,
                % const,false,private,this,while,constexpr,float,protected,thread_local,
                % const_cast,for,public,throw,std},
        morekeywords={syms,simplify,subs,double},
        emphstyle=\color{CPPViolet}, 
}
\hypersetup{colorlinks} % For colored hyperlinks             

\begin{document}
\maketitle
\section{配置环境}\marginnote[-0.5cm]{7月1日至2日完成}
% \par 配置虚拟专用网络。系统设置—网络代理更改为“手动”后,执行
% \begin{lstlisting}
%     cd /home/firefly/clash
%     ./clash -d . 
% \end{lstlisting}
% 可开启网络代理。已为git设置网络代理(仅在使用网络代理时,\lstinline{git clone}等命令方可生效。)。若想\textcolor{red}{取消},命令如下:
% \begin{lstlisting}
%     git config --global --unset http.proxy
%     git config --global --unset https.proxy
% \end{lstlisting}
\par 参考 \href{https://github.com/Intelligent-Quads/iq_tutorials}{Intelligent-Quads} 项目,配置开发环境。
\sidenote{Ubuntu 18.04, ROS Melodic, Gazebo 9, Ardupilot, MAVProxy, MAVROS}
\section{初步了解}\marginnote[-0.5cm]{7月2日至7月10日完成}
\par 认识无人机系统的整体架构与开发方式,掌握相关的基础概念。\sidenote{比如
    \begin{itemize}
        \item GCS(Ground Control Station)是什么
        \item FCU(Flight Control Unit)是什么
        \item MAV(Micro Air Vehicle)Link与MAVROS, MAVProxy有什么关系
        \item SITL和HITL的对比(Simulation/Hardware In the Loop)
    \end{itemize}    
}
\par 参考 \href{https://www.youtube.com/channel/UCuZy0c-uvSJglnZfQC0-uaQ}{Drone Software Development Tutorials} 系列视频,运行例程。
\section{成果}
\par 可通过MAVROS配置参数,屏蔽底层FCU的飞控算法,直接用无线电控制四旋翼无人机的每个电机的PWM占空比,为进一步的强化学习算法实践铺垫。\marginnote[-0.5cm]{7月11日至7月15日完成}
\sidenote{已将配置参数的过程写为\lstinline{.sh}脚本,运行后立刻生效。}
\newpage
\section{注意事项}
\subsection{关于MAVROS}
\begin{itemize}
    \item MAVROS的\;\lstinline{mav.parm}\;的数据格式与MAVProxy不同,前者使用逗号分隔,后者使用空格分隔。
    \item 无法同时调用\;\lstinline{rosrun mavros mavparam set TARGET VALUE}\;指令设置多个参数,建议延时后依次执行。
\end{itemize}
\subsection{关于QGroundControl}
\begin{itemize}
    \item 若运行多个无人机实例,需在Common Link中配置与MAVProxy共享的TCP连接。
\end{itemize}
% \printindex
\end{document}
